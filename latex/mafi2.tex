\documentclass{article}[12pt]

\usepackage{graphicx}
\usepackage[slovene]{babel}
\usepackage{psfrag}
\usepackage{epsf}
\usepackage{amsmath,amsfonts,amssymb,latexsym}
\usepackage{enumitem}
\usepackage[width=175mm,height=260mm,left=20mm,foot=10mm]{geometry}
\usepackage[utf8]{inputenc}
\usepackage{setspace}
%\doublespacing


\newtheorem{izrek}{Izrek}
\newtheorem{posledica}[izrek]{Posledica}
\newtheorem{lema}[izrek]{Lema}
\newtheorem{trditev}[izrek]{Trditev}
\newtheorem{domneva}[izrek]{Domneva}
\newtheorem{problem}[izrek]{Problem}
\newtheorem{vprasanje}[izrek]{Vprasanje}
\newtheorem{definicija}[izrek]{Definicija}
\newtheorem{opomba}[izrek]{Opomba}


\title{Matematična Fizika 2, domača naloga:\\
Polje gnezda nanocevk v valjasto omejenem prostoru}

     \author{
     \textsc{Rok Mihevc} \\[0.25em]
     {\small{Fakulteta za matematiko in fiziko}} \\[-0.25em]
     {\small{Univerza v Ljubljani}} \\[-0.25em]
     {\small\texttt{rok@mihevc.org}}
     }

\date{\today}



%%%%%%%%%%%%%%%%%%%%%%%%%%%%%%%%%%%%%%%%%%%%%%%%%%%%%%%%
%
% Author's definitions


\newenvironment{dokaz}%
{\noindent{\bf Dokaz.}\ }%
{\hfill$\Box$\par\bigskip}%

%%%%%%%%%%%%%%%%%%%%%%%%%%%%%%%
% mno?ice in funkcije
%%%%%%%%%%%%%%%%%%%%%%%%%%%%%%%
\newcommand{\dfnc}[3]{#1:#2\rightarrow #3}
\newcommand{\dset}[2]{\left\{#1 \:|\: #2\right\}}
\newcommand{\lset}[2]{\left\{#1, \ldots, #2\right\}}

%%%%%%%%%%%%%%%%%%%%%%%%%%%%%%%
% ?tevilske mno?ice
%%%%%%%%%%%%%%%%%%%%%%%%%%%%%%%

\newcommand{\NN}{\mathbb N}
\newcommand{\ZZ}{\mathbb Z}
\newcommand{\QQ}{\mathbb Q}
\newcommand{\RR}{\mathbb R}

%%%%%%%%%%%%%%%%%%%%%%%%%%%%%%%
% posebne prilagoditve ukazov
%%%%%%%%%%%%%%%%%%%%%%%%%%%%%%%
\newcommand{\cali}[1]{{\cal #1}}
\def\thx{\vartheta}
\def\epx{\varepsilon}
\def\rhx{\varrho}
\def\phx{\varphi}

\newcommand{\textem}[1]{{\sl #1}}

\renewcommand\a{\alpha}
\renewcommand\b{\beta}
\renewcommand\d{\delta}
\newcommand\D{\Delta}

\begin{document}

\maketitle


\section{Uvod}

Nanocevke so zelo uporabne kot izviri za hladno emisijo elektronov, ker ob njihovi površini že z majhno napetostjo dosežemo velike jakosti električnega polja. Ker je težko izolirati posamezno nanocevko, dostikrat uporabijo kar celo gnezdo cevk, kakor same zrastejo.\\
Kako se zaradi sosed polje med cevko in oddaljeno ploščato anodo oslabi? Vzemi poenostavljen primer, v katerem obravnavamo samo vrh polkroglasto zaključene cevke – torej kot majhno kroglo proti ravnini. Sosedne cevke predstavimo z mrežo krogel, tako da z zrcaljenjem nastane problem krogle v valjasto omejenem prostoru (z robnim pogojem II. vrste na mejnih ploskvah.

\section{Postopek}

\subsection{Numerični pristop}
Nalogo zastavimo kot reševanje parcialne diferencialne enačbe z robnimi pogoji prvega (Dirichlet) in
drugega reda (von Neumann) ter jo lahko rešujemo numerično. Zaradi cilindrične geometrije se 
poslužimo cilindričnih koordinat.
Naelektreno kroglo in ozemljeno ploščo opišemo z Robnimi pogoji 1.:
Na ozemljeni plošči je potencial enak 0, na krogli pa U0.
Slinice se ozemljene plošče in nabite krogle dotikajo pravokotno. 
Poenostavljen sistem naelektrene krogle s sosedami (Slika 2. in 3.) pa opišemo z Robnimi pogoji 2.:
Na ozemljeni plošči je potencial enak 0, na krogli pa U0.
Slinice se ozemljene plošče in nabite krogle dotikajo pravokotno. Na obodnem robu valjasto 
omejenega prostora pa so vzporedne z mejno ploskvijo.
Zaradi simetrije cilindrične koordinate theta, ki nam jo da poenostavljen problem (Slika 2.) ga lahko 
rešujemo v 2D ravnini, ki jo oklepata cilindrična vektorja z in r.

\begin{figure}[h]
\begin{center}
\includegraphics[width=5cm]{slike/rezultat1.jpg}
\caption{Vključite lahko tudi sliko.}
\end{center}
\end{figure}

\subsection{Prvi podrazdelek prvega razdelka}

Tu napišete vsebino prvega podrazdelka prvega razdelka. \LaTeX je namenjen predvsem matematičnim besedilom. Vsaka matematična formula, tudi zgolj $x$, se pojavi med dvema \$ znakoma: \verb.$x$.. Seveda pa lahko napišemo tudi bolj zapletene izraze: $$\int_0^\infty \frac{1}{x^3}dx.$$


\begin{izrek}
\label{iz:prviIzrek}
Tu napišete besedilo izreka.
\end{izrek}
\begin{dokaz}
Tu napišete besedilo dokaza. Lahko je dolgo več vrstic, ali pa sem in tja preide tudi v nov odstavek.

Nov odstavek se začne po prazni vrstici.
\end{dokaz}


Sledi še primer naštevanja z možnim sklicevanjem, npr. na Izrek \ref{iz:prviIzrek}: 
\begin{enumerate}[label=(\roman{*}), ref=(\roman{*})]
\item \label{it:prva} Prva alineja je pred alinejo \ref{it:druga}.
\item \label{it:druga} Druga alineja sledi alineji \ref{it:prva}.
\end{enumerate}

\begin{thebibliography}{99}



\bibitem{oznakaKnjige} I. Priimek, Naslov knjige, Zalo?nik, MestoIzdaje, Letnica.

\bibitem{oznakaClanka} I. Priimek, Naslov ?lanka, Naslov revije Letnik (Leto) str.\ Od--Do.
\end{thebibliography}


\end{document}
