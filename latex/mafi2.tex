\documentclass{article}[12pt]

\usepackage{graphicx}
\usepackage[slovene]{babel}
\usepackage{psfrag}
\usepackage{epsf}
\usepackage{amsmath,amsfonts,amssymb,latexsym}
\usepackage{enumitem}
\usepackage[width=175mm,height=260mm,left=20mm,foot=10mm]{geometry}
\usepackage[utf8]{inputenc}
\usepackage{setspace}
\usepackage{epstopdf}
%\doublespacing


\newtheorem{izrek}{Izrek}
\newtheorem{posledica}[izrek]{Posledica}
\newtheorem{lema}[izrek]{Lema}
\newtheorem{trditev}[izrek]{Trditev}
\newtheorem{domneva}[izrek]{Domneva}
\newtheorem{problem}[izrek]{Problem}
\newtheorem{vprasanje}[izrek]{Vprasanje}
\newtheorem{definicija}[izrek]{Definicija}
\newtheorem{opomba}[izrek]{Opomba}


\title{Matematična Fizika 2, domača naloga:\\
Polje gnezda nanocevk v valjasto omejenem prostoru}

     \author{
     \textsc{Rok Mihevc} \\[0.25em]
     {\small{Fakulteta za matematiko in fiziko}} \\[-0.25em]
     {\small{Univerza v Ljubljani}} \\[-0.25em]
     {\small\texttt{rok@mihevc.org}}
     }

\date{\today}



%%%%%%%%%%%%%%%%%%%%%%%%%%%%%%%%%%%%%%%%%%%%%%%%%%%%%%%%
%
% Author's definitions


\newenvironment{dokaz}%
{\noindent{\bf Dokaz.}\ }%
{\hfill$\Box$\par\bigskip}%

%%%%%%%%%%%%%%%%%%%%%%%%%%%%%%%
% mno?ice in funkcije
%%%%%%%%%%%%%%%%%%%%%%%%%%%%%%%
\newcommand{\dfnc}[3]{#1:#2\rightarrow #3}
\newcommand{\dset}[2]{\left\{#1 \:|\: #2\right\}}
\newcommand{\lset}[2]{\left\{#1, \ldots, #2\right\}}

%%%%%%%%%%%%%%%%%%%%%%%%%%%%%%%
% ?tevilske mno?ice
%%%%%%%%%%%%%%%%%%%%%%%%%%%%%%%

\newcommand{\NN}{\mathbb N}
\newcommand{\ZZ}{\mathbb Z}
\newcommand{\QQ}{\mathbb Q}
\newcommand{\RR}{\mathbb R}

%%%%%%%%%%%%%%%%%%%%%%%%%%%%%%%
% posebne prilagoditve ukazov
%%%%%%%%%%%%%%%%%%%%%%%%%%%%%%%
\newcommand{\cali}[1]{{\cal #1}}
\def\thx{\vartheta}
\def\epx{\varepsilon}
\def\rhx{\varrho}
\def\phx{\varphi}

\newcommand{\textem}[1]{{\sl #1}}

\renewcommand\a{\alpha}
\renewcommand\b{\beta}
\renewcommand\d{\delta}
\newcommand\D{\Delta}

\begin{document}

\maketitle


\section{Uvod}

Nanocevke so zelo uporabne kot izviri za hladno emisijo elektronov, ker ob njihovi površini že z majhno napetostjo dosežemo velike jakosti električnega polja. Ker je težko izolirati posamezno nanocevko, dostikrat uporabijo kar celo gnezdo cevk, kakor same zrastejo.\\
Kako se zaradi sosed polje med cevko in oddaljeno ploščato anodo oslabi? Vzemi poenostavljen primer, v katerem obravnavamo samo vrh polkroglasto zaključene cevke – torej kot majhno kroglo proti ravnini. Sosedne cevke predstavimo z mrežo krogel, tako da z zrcaljenjem nastane problem krogle v valjasto omejenem prostoru (z robnim pogojem II. vrste na mejnih ploskvah - Slika \ref{shema}).

\begin{figure}[h]
\begin{center}
\includegraphics[width=13cm]{slike/shema.eps}
\caption{Shema približka polja nanocevk}
\label{shema}
\end{center}
\end{figure}

\section{Postopek}

\subsection{Opis fizikalnega problema}
Nalogo zastavimo kot reševanje parcialne diferencialne enačbe z robnimi pogoji prvega (Dirichletov problem) in
drugega reda (von Neumannov problem) ter jo lahko rešujemo numerično. Zaradi izbrane poenostavitve problema se 
poslužimo cilindričnih koordinat. \newpage
Naelektreno kroglo in ozemljeno ploščo (Slika \ref{layout-1}) opišemo z \textit{Robnimi pogoji 1.}:\newline

$ \phi (z = D) = 0 $

$ \phi (r^2 + z^2 = r_o^2) = U_0 $

$ \frac{\partial \phi}{\partial \vec{n}} (r^2 + z^2 = r_0^2) = k $
\newline

Na ozemljeni plošči je potencial enak 0, a pričakujemo inducirano napetost. Kroglo postavimo na potencial $U_0$.
Slinice se ozemljene plošče in nabite krogle dotikajo pravokotno.

\begin{figure}[h]
\begin{center}
\includegraphics[width=14cm]{slike/layout-1.eps}
\caption{Krogla na potencialu $U_0$ in ozemljena plošča}
\label{layout-1}
\end{center}
\end{figure}

Poenostavljen sistem naelektrene krogle s sosedami (Slika \ref{layout-2}) pa opišemo z \textit{Robnimi pogoji 2.}:\newline

$ \phi (z = D) = 0 $

$ \phi (r^2 + z^2 = r_o^2) = U_0 $

$ \frac{\partial \phi}{\partial \vec{n}} ( r^2 + z^2 = r_0^2) = k $

$ \frac{\partial \phi}{\partial \vec{n}} (r = d) = 0 $
\newline

Na ozemljeni plošči je potencial enak 0, a pričakujemo inducirano napetost. Kroglo postavimo na potencial $U_0$.
Slinice se ozemljene plošče in nabite krogle dotikajo pravokotno. Na obodu valjasto omejenega prostora pa so vzporedne z mejno ploskvijo.

\newpage

\begin{figure}[h]
\begin{center}
\includegraphics[width=14cm]{slike/layout-2.eps}
\caption{Množica krogel na potencialu $U_0$ in ozemljena plošča}
\label{layout-2}
\end{center}
\end{figure}

Zaradi simetrije cilindrične koordinate $\theta$, problem lahko rešujemo v ravnini, ki jo oklepata cilindrična vektorja $\vec{z}$ in $\vec{r}$ (Slika \ref{layout-3}).

\begin{figure}[h]
\begin{center}
\includegraphics[width=13cm]{slike/layout-3.eps}
\caption{Končna poenostavitev problema}
\label{layout-3}
\end{center}
\end{figure}

\subsection{Numerično reševanje problema}

Ker se je analitični pristop izkazal za netrivialnega, sem se problema lotil numerično.
Izbral sem orodje \textit{assempde} (iz zbirke orodij Matlab), ki ob dani geometriji in robnih pogojih z metodo končnih elementov numerično izračuna rešitev parcialne diferencialne enačbe.
Zaradi fizikalne smiselnosti sem v numeričen postopek dodal še dva robna pogoja:

$ \phi (z \to \inf) = U_0 $ - cevke so pritrjene na nekakšno elektrodo, ki jo uporabimo za vzdrževanje potenciala

$ \frac{\partial \phi}{\partial \vec{n}} (z \to \inf) = k $ - elektroda je prevodnik

\section{Rezultati}

Rezultat so grafi prereza električnega polja fizikalnega sistema pod različnimi koti (Slika: \ref{slika01-koti}) in slike jakosti električnega polja celotnega sistema. Pri izračunih sem variiral premer sfere proti premeru valja.

\begin{center}
\includegraphics[width=15cm]{slike/slika001.png}
\includegraphics[width=15cm]{slike/presek001.png}
\end{center}

\begin{center}
\includegraphics[width=15cm]{slike/slika01.png}
\includegraphics[width=15cm]{slike/presek01.png}
\end{center}

\begin{center}
\includegraphics[width=15cm]{slike/slika05.png}
\includegraphics[width=15cm]{slike/presek05.png}
\end{center}

\begin{center}
\includegraphics[width=15cm]{slike/slika09.png}
\includegraphics[width=15cm]{slike/presek09.png}
\end{center}

\newpage
\begin{figure}
\begin{center}
\includegraphics[width=15cm]{slike/slika01-koti.png}
\caption{Primer slike polja, z oznakami prikazanih prerezov}
\label{slika01-koti}
\end{center}
\end{figure}


\section{Zaključek}

Ko je konica nanocevke velika napram razdalji do svojih sosed, je polje precej manj homogeno kot če je konica majhna napram razdalji do svojih sosed.
Iz grafov prereza polja lahko vidimo tudi, da prerez polja ob elektrodi ohranja obliko, a se njegov gradient s povečevanjem sfere veča.


\begin{thebibliography}{99}

\bibitem{kodre} I. Kuščer in A. Kodre: Matematika v fiziki in tehniki, DMFA, Ljubljana, 1994
\bibitem{matlabDoc} Dokumentacija programa Matlab - http://www.mathworks.com/help/techdoc/

%  I. Priimek, Naslov knjige, Zalo?nik, MestoIzdaje, Letnica.
% \bibitem{oznakaClanka} I. Priimek, Naslov ?lanka, Naslov revije Letnik (Leto) str.\ Od--Do.
\end{thebibliography}


% \subsection{Prvi podrazdelek prvega razdelka}

% Tu napišete vsebino prvega podrazdelka prvega razdelka. \LaTeX je namenjen predvsem matematičnim besedilom. Vsaka matematična formula, tudi zgolj $x$, se pojavi med dvema \$ znakoma: \verb.$x$.. Seveda pa lahko napišemo tudi bolj zapletene izraze: $$\int_0^\infty \frac{1}{x^3}dx.$$


% \begin{izrek}
% \label{iz:prviIzrek}
% Tu napišete besedilo izreka.
% \end{izrek}
% \begin{dokaz}
% Tu napišete besedilo dokaza. Lahko je dolgo več vrstic, ali pa sem in tja % preide tudi v nov odstavek.

%Nov odstavek se začne po prazni vrstici.
%\end{dokaz}


% Sledi še primer naštevanja z možnim sklicevanjem, npr. na Izrek \ref{iz:prviIzrek}: 
% \begin{enumerate}[label=(\roman{*}), ref=(\roman{*})]
% \item \label{it:prva} Prva alineja je pred alinejo \ref{it:druga}.
% \item \label{it:druga} Druga alineja sledi alineji \ref{it:prva}.
% \end{enumerate}


\end{document}
